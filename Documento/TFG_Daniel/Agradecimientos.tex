\textbf{\Huge{Agradecimientos}} \\

\vspace*{10mm}

TITULACIÓN: Graduado en Ingeniería Aeroespacial

DEPARTAMENTO: Materiales y Producción Aeroespacial 

TIPO DE TFG: Especial

INTENSIFICACIONES A LAS QUE SE OFERTA: Todas

TÍTULO DEL TRABAJO: Detección y análisis de impactos sobre estructuras de material compuesto con sensores integrados\\

\textbf{CONTENIDOS Y OBJETIVOS DEL TRABAJO:}

En el marco de las estructuras inteligentes, las estructuras con sensores integrados permiten la determinación de la existencia de daños o de eventos mediante el tratamiento de las señales de los sensores, en lo que se conoce como SHM (Structural Health Monitoring). En este marco, se va a trabajar con una red de sensores PZT integrados en una placa de material compuesto con y sin rigidizadores. Se analizarán los impactos realizados sobre la misma y se estudiará su respuesta para tratar de discriminar no solo la posición del impacto, sino también discriminar la energía del mismo. Adicionalmente se tratará de discriminar entre la masa y la velocidad del impacto y, en caso de que se produzcan daños en la misma (delaminaciones, roturas, etc) cuantificarlos y caracterizarlos. 

Con este objetivo, se desarrollarán redes neuronales (ANN-Artificial Neural Networks) que serán entrenadas con resultados experimentales y teóricos. Las ANN son unas excelentes clasificadoras y analizadoras de tendencias, por lo que se propone su uso para el estudio de las señales debidas al impacto.

Este TFG presenta un carácter marcadamente interdisciplinar, ya que abarca desde las estructuras de material compuesto, adquisición de medida mediante sensores y tratamiento de datos.
